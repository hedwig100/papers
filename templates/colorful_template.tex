\documentclass[uplatex]{jsarticle}

%% Packages
\usepackage[dvipdfmx]{graphicx,color,hyperref}
\usepackage{algorithm}
\usepackage{algorithmic}
\usepackage{url}
\usepackage{lscape}
\usepackage{mathtools}
\usepackage{here}
\usepackage{amsmath,amssymb,amsfonts}
\usepackage{amsthm}
\usepackage{tikz}
\usepackage{tcolorbox}
\usepackage{pxjahyper}

%% Theorem Styles
\newtheorem{theorem}{定理}
\newtheorem{proposition}{命題}
\newtheorem{cor}{系}
\newtheorem{definition}{定義}
\newtheorem{problem}{問題}
\theoremstyle{remark}
\newtheorem{remark}{注意}
\newtheorem{requirement}{条件}

%% Environment (Colorful Box)
\newenvironment{method}[1]{
    \begin{tcolorbox}[
        colframe=green!50!black,
        colback=green!50!black!10!white,
        colbacktitle=green!50!black!40!white,
        coltitle=black,
        fonttitle=\bfseries,
        title={#1}
    ]
}{
    \end{tcolorbox}
}

\newenvironment{experiment}[1]{
    \begin{tcolorbox}[
        colframe=violet,
        colback=violet!10!white,
        colbacktitle=violet!40!white,
        coltitle=black,
        fonttitle=\bfseries,
        title={#1}
    ]
}{
    \end{tcolorbox}
}

%% Title
\title{タイトル}
\author{\empty}
\date{\empty}

%% Document body
\begin{document}
\maketitle

\section{概要}
\begin{tcolorbox}[fonttitle=\bfseries]
ここに概要を書く
\end{tcolorbox}

\section{手法}
\begin{method}{手法1}
\begin{align*}
    1 + 1 = 2
\end{align*}
\end{method}

\section{実験}
\begin{experiment}{実験1}
\ref{fig:template}のように図を引用することができる.
\end{experiment}

\begin{figure}
    \centering
    \includegraphics[width=0.5\textwidth]{img/image.png}
    \caption{キャプション}
    \label{fig:template}
\end{figure}

\section{感想}
\begin{tcolorbox}[colframe=brown,
  colback=brown!10!white,
  colbacktitle=brown!40!white,
  coltitle=black,fonttitle=\bfseries]
\cite{mypaper}すごい
\end{tcolorbox}


\bibliographystyle{jplain}
\bibliography{template.bib}

\end{document}