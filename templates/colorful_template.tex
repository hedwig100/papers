\documentclass[uplatex]{jsarticle}

%% Packages
\usepackage[dvipdfmx]{graphicx,color,hyperref}
\usepackage[dvipsnames]{xcolor}
\usepackage{algorithm}
\usepackage{algorithmic}
\usepackage{url}
\usepackage{lscape}
\usepackage{mathtools}
\usepackage{here}
\usepackage{amsmath,amssymb,amsfonts}
\usepackage{amsthm}
\usepackage{tikz}
\usepackage{tcolorbox}
\usepackage{pxjahyper}

%% Theorem Styles
\newtheorem{theorem}{定理}
\newtheorem{proposition}{命題}
\newtheorem{cor}{系}
\newtheorem{definition}{定義}
\newtheorem{problem}{問題}
\theoremstyle{remark}
\newtheorem{remark}{注意}
\newtheorem{requirement}{条件}

%% Environment (Colorful Box)
\newenvironment{qbox}[1]{
    \begin{tcolorbox}[
        colframe = RoyalBlue,
        colback = RoyalBlue!10!White,
        title = {#1},
        fonttitle = \bfseries,
        breakable = true
    ]
}{
    \end{tcolorbox}
}

%% Title
\title{タイトル}
\author{\empty}
\date{\empty}

%% Document body
\begin{document}
\maketitle

\begin{tcolorbox}[colback=yellow!10, colframe=orange, title=テスト]
これは tcolorbox のテストです。
\end{tcolorbox}

% \begin{qbox}{手法}
%     \begin{align*}
%       1 + 1 = 2
%     \end{align*}
% \end{qbox}

% \begin{qbox}{実験}
%   \ref{fig:template}のように図を引用することができる.
%   \begin{figure}
%     \centering
%     \includegraphics[width=0.5\textwidth]{img/image.png}
%     \caption{キャプション}
%     \label{fig:template}
%   \end{figure}
% \end{qbox}

% \begin{qbox}{感想}
% すごい。
% \end{qbox}



\bibliographystyle{jplain}
\bibliography{template.bib}

\end{document}