\documentclass[dvipdfmx,uplatex]{jsarticle}

%% Packages
\usepackage{graphicx,color,hyperref}
\usepackage{algorithm}
\usepackage{algorithmic}
\usepackage{url}
\usepackage{lscape}
\usepackage{mathtools}
\usepackage{here}
\usepackage{amsmath,amssymb,amsfonts}
\usepackage{amsthm}
\usepackage{tikz}
\usepackage{tcolorbox}
\usepackage{pxjahyper}

%% Theorem Styles
\newtheorem{theorem}{定理}
\newtheorem{proposition}{命題}
\newtheorem{cor}{系}
\newtheorem{definition}{定義}
\newtheorem{problem}{問題}
\theoremstyle{remark}
\newtheorem{remark}{注意}
\newtheorem{requirement}{条件}

%% Environment (Colorful Box)
\newenvironment{simplebox}{
    \begin{tcolorbox}[
        fonttitle=\bfseries,
    ]
}{
    \end{tcolorbox}
}

\newenvironment{method}[1]{
    \begin{tcolorbox}[
        colframe=green!50!black,
        colback=green!50!black!10!white,
        colbacktitle=green!50!black!40!white,
        coltitle=black,
        fonttitle=\bfseries,
        title={#1}
    ]
}{
    \end{tcolorbox}
}

\newenvironment{experiment}[1]{
    \begin{tcolorbox}[
        colframe=violet,
        colback=violet!10!white,
        colbacktitle=violet!40!white,
        coltitle=black,
        fonttitle=\bfseries,
        title={#1}
    ]
}{
    \end{tcolorbox}
}

\newenvironment{kansou}{
    \begin{tcolorbox}[
        colframe=brown,
        colback=brown!10!white,
        colbacktitle=brown!40!white,
        coltitle=black,fonttitle=\bfseries
    ]
}{
    \end{tcolorbox}
}

%% Title
\title{Sphinteract: Resolving Ambiguities in NL2SQL Through User Interaction}
\author{\empty}
\date{\empty}

%% Document body
\begin{document}
\maketitle

\begin{itemize}
    \item Link: \url{https://www.vldb.org/pvldb/vol18/p1145-zhao.pdf}
    \item Conference: VLDB2025
    \item Citation:
    \item Arxiv:
\end{itemize}

\section{概要}
\begin{simplebox}
\begin{itemize}
    \item NL2SQLにおいて、ユーザが発する質問の曖昧性を解消するためのフレームワークSpinteractを提案する。これはあいまいさを解消するためにユーザフィードバックを対話的に取り入れる。
    \item ユーザフィードバックを取り入れるための質問はSummarize, Review, Askという3つの方針のどれかに基づいて生成される(SRAパラダイム)。
    \item SpinteractによってKaggleDBQA、BIRDによる実験で最大で42\%の精度向上が達成できることを示した。
\end{itemize}
\end{simplebox}

\section{問題設定}
\begin{simplebox}
\begin{itemize}
    \item 解答の流れと定式化
    \begin{itemize}
        \item ユーザの自然言語の質問に対して、SQLクエリを生成する。このクエリが誤っている場合は対話プロセスが開始され、ユーザに質問が提示される。ユーザの回答を受け取った後、SQLクエリが修正される。このプロセスは正しいSQLクエリが生成されるか、停止基準が満たされるまで繰り返される。
        \item この条件の下で、ラウンド数$n$とLLM呼び出しのコストに依存するあるコスト関数を最小化するというように定式化される。
    \end{itemize}
    \item 求められる条件として
    \begin{itemize}
        \item ユーザへの質問はSQLがわからない人にとっても理解可能であること
        \item ユーザへの質問への解答は簡単であること: 選択肢方式など
        \item 無限に質問ができるわけではないため、不要なやりとりを避ける必要がある: 固定回数の方式と、あいまいさが検出されなくなった時点で停止する方式の両方を検討する
        \item ユーザに実行可能なクエリが提示できること
    \end{itemize}
\end{itemize}
\end{simplebox}

\section{自然言語の質問に曖昧性があることの検証}
\begin{simplebox}
\begin{itemize}
    \item KaggleDBQAからランダムに64この質問を抽出し、これらの質問に曖昧性があることを検証する。具体的には次のようにした。
    \item 複数人に選んだ質問のSQLを作成してもらい、これらの実行結果がどの程度一致するかを検証した。
    \item その結果64問中8\%しかそれは一致しなかった。また最も可能性の高い回答の割合についても調べたが、およそ50\%の質問しか多数派(50\%を超える)の回答が存在しなかった。
\end{itemize}
上の結果をもとに、あいまい性の種類を分類した。
\begin{itemize}
    \item AmbColumn: 質問内のエンティティがデータスキーマと明確に対応付けられない場合(e.g. 似たカラムが複数存在する、カラムが不明瞭など)
    \item AmbOutput: 出力テーブルの形式や並び順が指定されていない場合
    \item AmbQuestion: 質問が漠然としている場合
    \item AmbValue: 使用すべき述語値に不確実性がある場合に発生する(たとえば数値であるべき値が文字列として与えられた場合など)
\end{itemize}
\end{simplebox}

\begin{figure}
    \centering
    \includegraphics[width=0.7\textwidth]{img/sphinteract/overview.png}
    \caption{Sphinteractの概要}
    \label{fig:overview}
\end{figure}

\section{手法}
\begin{method}{Sphinteract}
\begin{itemize}
    \item Sphinteractフレームワークはユーザとのマルチラウンドの対話で構成される、このプロセスは次の二つの部分に分けられる(1. SQLクエリの生成、2. ユーザとの対話)。図\ref{fig:overview}に示される。
    \begin{itemize}
        \item まず自然言語とデータベーススキーマのみを用いてSQLクエリを生成する。
        \item このクエリを実行し、出力とクエリをユーザに提示する。ユーザはこのクエリが期待に添わない場合に、その旨をフィードバックする。
        \item フィードバックをもとに複数選択式質問を生成し、ユーザに提示してFBを得る。
        \item このSQLクエリと実行結果をユーザに提示する。
        \item このプロセスを終了条件が満たされるまで続ける。
    \end{itemize}
    \item SQL生成はDAIL-SQLのプロンプトを用いる。
    \item CQ: 曖昧性解消のための質問の生成のプロンプトには誤ったクエリとそのFBを入力として使用する。ここではSRA(Summarize, Review, Ask)という新手法を導入する。すなわち、これまでの対話から情報を要約し、残る曖昧性を評価し、あいまい性解消のための質問をユーザに提示する。
    \item ES: すべての曖昧性を除去した場合でもSQLを生成できない場合があり、この場合に無限にこのプロセスが続くことを防ぐために、Early Stoppingができるようにする(曖昧性がない場合には終了できることをプロンプトに書いておく)。
\end{itemize}
\end{method}


\section{実験結果}
\begin{experiment}{実験手法}
\begin{itemize}
    \item LLM設定
    \begin{itemize}
        \item GPT3.5 Turbo, GPT-4 Turbo
        \item 温度パラメータは$0.0$
    \end{itemize}
    \item ベースライン
    \begin{itemize}
        \item DAIL-SQL
        \item ファインチューニングモデルや非標準情報を用いたプロンプトは比較対象外
    \end{itemize}
    \item フィードバックオラクル
    \begin{itemize}
        \item ユーザとの質問をシミュレートするオラクルを使用して実験を行う
        \item オラクルの入力: 確認質問+ゴールデンSQL
    \end{itemize}
    \item Few Shotの例
    \begin{itemize}
        \item 確認の質問生成用に: 静的に8例を選択
        \item Simple Feedbackの場合のSQL生成: 静的に8例を選択
        \item CQのあとのSQL生成に: ユーザ質問と問題集内の質問のベクトル埋め込み間のコサイン類似度に基づいて選択 
    \end{itemize}
\end{itemize}
\end{experiment}

\begin{figure}
    \centering
    \includegraphics[width=\textwidth]{img/sphinteract/zero-shot.png}
    \caption{ゼロショットの場合の実験結果}
    \label{fig:zero-shot}
\end{figure}

\begin{figure}
    \centering
    \includegraphics[width=\textwidth]{img/sphinteract/few-shot.png}
    \caption{Fewショットの場合の実験結果}
    \label{fig:few-shot}
\end{figure}

\begin{experiment}{実験結果}
\begin{itemize}
    \item Zero Shotの場合(図\ref{fig:zero-shot})
    \begin{itemize}
        \item Simple Feedback: よいか悪いかだけのフィードバックなので精度向上は小さい
        \item Clarification Questions: KaggleDBQAでは42.3\%の精度向上、BIRDでは26.9\%の精度向上
        \item CQ + Early Stopping: すべての曖昧さが解消されたと判断された時点で対話を終了するため、対話回数は少なくなり、精度向上も限定的
        \item ただし、CQでは対話回数が多くなりコストがCQ+ESより高くなる。
    \end{itemize}
    \item Few Shotの場合(図\ref{fig:few-shot})
    \begin{itemize}
        \item Simple Feedback: 例の数をある程度増やすと精度が微減または横ばいになる
        \item CQ: SQL例が多くなるほど精度が向上する
        \item CQ+ES: 例の数をある程度増やすと精度が微減または横ばいになる
        \item いずれのケースでもZero-Shotと比べた場合の精度改善は大幅(Simple Feedbackでも大きく改善する)。
    \end{itemize}
    \item Userスタディ
    \begin{itemize}
        \item 実際に11名のデータサイエンティストによるユーザスタディを行った
        \item 90\%のユーザは質問に正しく解答してくれたと答え、認知負荷も小さかった。
        \item 平均質問回数は2.18回となった
        \item FBとしては「自由形式FBの方が良い場合もある、AmbValueに関する曖昧性の解決をより多く解決してほしい」などが挙げられた。
    \end{itemize}
\end{itemize}
\end{experiment}

\section{感想}
\begin{kansou}
\begin{itemize}
  \item ユーザの質問に曖昧性がある場合にインタラクティブにユーザ意図を確認することでNL2SQLの精度向上を目指していて、現実に近い環境を考慮している点が面白いと思った。
  \item FewShotの例の選択がどのSQLを生成するタイミング化で用いる例の選び方が変わっていてそこを変える理由は何だろうと思った。
  \item 単純な二値フィードバックでもFewShotを用いると大きく精度が向上するのは興味深い。
\end{itemize}
\end{kansou}
% \bibliographystyle{jplain}
% \bibliography{template.bib}

\end{document}