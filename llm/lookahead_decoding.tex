\documentclass[uplatex]{jsarticle}

%% Packages
\usepackage[dvipdfmx]{graphicx,color,hyperref}
\usepackage{algorithm}
\usepackage{algorithmic}
\usepackage{url}
\usepackage{lscape}
\usepackage{mathtools}
\usepackage{here}
\usepackage{amsmath,amssymb,amsfonts}
\usepackage{amsthm}
\usepackage{tikz}
\usepackage{tcolorbox}
\usepackage{pxjahyper}

%% Theorem Styles
\newtheorem{theorem}{定理}
\newtheorem{proposition}{命題}
\newtheorem{cor}{系}
\newtheorem{definition}{定義}
\newtheorem{problem}{問題}
\theoremstyle{remark}
\newtheorem{remark}{注意}
\newtheorem{requirement}{条件}

%% Environment (Colorful Box)
\newenvironment{method}[1]{
    \begin{tcolorbox}[
        colframe=green!50!black,
        colback=green!50!black!10!white,
        colbacktitle=green!50!black!40!white,
        coltitle=black,
        fonttitle=\bfseries,
        title={#1}
    ]
}{
    \end{tcolorbox}
}

\newenvironment{experiment}[1]{
    \begin{tcolorbox}[
        colframe=violet,
        colback=violet!10!white,
        colbacktitle=violet!40!white,
        coltitle=black,
        fonttitle=\bfseries,
        title={#1}
    ]
}{
    \end{tcolorbox}
}

%% Title
\title{Break the Sequential Dependency of LLM Inference Using Lookahead Decoding}
\author{\empty}
\date{\empty}

%% Document body
\begin{document}
\maketitle

\begin{itemize}
    \item Link: \url{https://dl.acm.org/doi/10.5555/3692070.3692631}
    \item Conference: ICML2024
\end{itemize}

\section{概要}
\begin{tcolorbox}[fonttitle=\bfseries]
ここに概要を書く
\end{tcolorbox}

\section{手法}
\begin{method}{手法1}
\begin{align*}
    1 + 1 = 2
\end{align*}
\end{method}

\section{実験}
\begin{experiment}{実験1}
\ref{fig:template}のように図を引用することができる.
\end{experiment}

\begin{figure}
    \centering
    \includegraphics[width=0.5\textwidth]{img/image.png}
    \caption{キャプション}
    \label{fig:template}
\end{figure}

\section{感想}
\begin{tcolorbox}[colframe=brown,
  colback=brown!10!white,
  colbacktitle=brown!40!white,
  coltitle=black,fonttitle=\bfseries]
\cite{mypaper}すごい
\end{tcolorbox}

\section{MEMO}
ここはコンパイルするときにコメントアウトする!!
\begin{itemize}
    \item Attention特にCausal attentionの説明
    \item Autoregressive LLMの動き方
    \item Jacobi decoding -> あまりうまくいかない、違う場所にtokenが置かれることが多い
\end{itemize}
\subsection{Abstract}
大規模言語モデル(LLM)の自己回帰デコーディングは、メモリ帯域幅によって制限されており、これが高いレイテンシ(遅延)と、最新のアクセラレータの並列処理能力の著しい浪費につながっています。LLMのデコーディングを高速化するための既存の手法(例:投機的デコーディング)は、ドラフトモデル(補助モデル)を必要としますが、このモデルの入手は容易ではなく、汎用性もありません。

本論文では、LOOKAHEAD DECODINGという、補助モデルやデータストアを必要とせずにLLMのデコーディングを高速化する、正確で並列なデコーディングアルゴリズムを提案します。このアルゴリズムは、ステップあたりの計算量(log(FLOPs))を増やすことで、デコーディングの総ステップ数を削減することを可能にします。また、単一または複数の最新のアクセラレータ上でより並列化が可能であり、メモリ効率の良い注意機構(例:FlashAttention)とも互換性があります。

我々のLOOKAHEAD DECODINGの実装は、MT-benchで自己回帰デコーディングを最大1.8倍、コード補完タスクでは複数GPUでの強力なスケーリングにより最大4倍高速化できることを示しました。

コードはこちらで公開されています: https://github.com/hao-ai-lab/LookaheadDecoding

\subsection{2 Background}
このセクションでは、自己回帰デコーディングとヤコビデコーディングの両方を、非線形システムを解くという観点から定式化します。

\subsubsection{デコーダモデルにおける因果的アテンション}

今日の大規模言語モデル (LLM) のほとんどは、主に2つの要素で構成されています。一つはトークン単位のモジュール(MLPや正規化層を含む)、もう一つはアテンションモジュールです。

アテンションモジュール内ではトークン同士が相互作用しますが、その他のトークン単位のモジュールでは、トークンは互いに情報をやり取りすることなく処理されます。

アテンション層は、クエリ (Q)、キー (K)、バリュー (V) という3つの入力要素で構成され、それぞれの i番目のトークンは
$Q_i, K_i, V_i$と表記されます。

アテンション層は次の演算を実行します: $O = softmax(QK^\top)V$.

デコーダモデル特有の因果的アテンションでは、$QK^\top$
  に下三角マスクが適用されます。これにより、$O_i$
  (出力$O$の$i$番目のトークン)は、$Q_i$
  と, $j \leq i$ である $K_j$
  および $V_j$
  からのみ計算されることが保証されます。

LLM内の他のすべての層がトークン単位の操作を実行するため、任意のモデル入力$x$と出力$o$に対して、$o_i$
  (出力$o$の$i$番目のトークン)は、$j\leq i$である$x_j$
  (入力$x$の$j$番目のトークン)によってのみ影響を受けます。

\subsubsection{LLMにおける自己回帰デコーディング}
これまでの出力と入力から一個ずつ推論する

\subsubsection{Guess-And-Verify Paradig}
一個ずつではなくて数個ずつ推論する

\subsubsection{Jacobi decoding}
Autoregressiveなdecodingを非線型方程式とみなして
それをJacobi法で解くことで解を求める。
Q微分できなくね??前の論文ちゃんと読まないとわからなさそう。

\subsubsection{Limitations of Jacobi decoding}

\subsection{3 Lookahead Decoding}
\begin{enumerate}
    \item Lookahead branch; jacobiを回す
    \item Verification branch; jacobiで得られたいくつかの候補がLLM distributionと一致するか確かめる
    \item Collect n-grams; いい感じのn-gramを選ぶ
    \item Update lookahead branch; 次ステップのためにbranchをupdate
\end{enumerate}

\bibliographystyle{jplain}
\bibliography{template.bib}

\end{document}