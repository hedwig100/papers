\documentclass[dvipdfmx,uplatex]{jsarticle}

%% Packages
\usepackage{graphicx,color,hyperref}
\usepackage{algorithm}
\usepackage{algorithmic}
\usepackage{url}
\usepackage{lscape}
\usepackage{mathtools}
\usepackage{here}
\usepackage{amsmath,amssymb,amsfonts}
\usepackage{amsthm}
\usepackage{tikz}
\usepackage{tcolorbox}
\usepackage{pxjahyper}

%% Theorem Styles
\newtheorem{theorem}{定理}
\newtheorem{proposition}{命題}
\newtheorem{cor}{系}
\newtheorem{definition}{定義}
\newtheorem{problem}{問題}
\theoremstyle{remark}
\newtheorem{remark}{注意}
\newtheorem{requirement}{条件}

%% Environment (Colorful Box)
\newenvironment{simplebox}{
    \begin{tcolorbox}[
        fonttitle=\bfseries,
    ]
}{
    \end{tcolorbox}
}

\newenvironment{method}[1]{
    \begin{tcolorbox}[
        colframe=green!50!black,
        colback=green!50!black!10!white,
        colbacktitle=green!50!black!40!white,
        coltitle=black,
        fonttitle=\bfseries,
        title={#1}
    ]
}{
    \end{tcolorbox}
}

\newenvironment{experiment}[1]{
    \begin{tcolorbox}[
        colframe=violet,
        colback=violet!10!white,
        colbacktitle=violet!40!white,
        coltitle=black,
        fonttitle=\bfseries,
        title={#1}
    ]
}{
    \end{tcolorbox}
}

\newenvironment{kansou}{
    \begin{tcolorbox}[
        colframe=brown,
        colback=brown!10!white,
        colbacktitle=brown!40!white,
        coltitle=black,fonttitle=\bfseries
    ]
}{
    \end{tcolorbox}
}

%% Title
\title{OmniSQL: Synthesizing High-quality Text-to-SQL Data at Scale}
\author{\empty}
\date{\empty}

%% Document body
\begin{document}
\maketitle

\begin{itemize}
    \item Link: \url{https://www.vldb.org/pvldb/vol18/p4695-li.pdf}
    \item Conference: VLDB 2025
    \item Citation:
    \item Arxiv: \url{https://arxiv.org/abs/2503.02240}
\end{itemize}

\section{概要}
\begin{simplebox}
\begin{itemize}
    \item Text2SQLタスクにおいてはクローズドモデルを使ってプロンプトで性能向上を図るか、QAデータセットを作成してオープンモデルをファインチューニングすることで性能向上を図ることができる。
    \item 本論文では、後者のアプローチに注目し、大規模な高品質Text2SQLデータセットを合成する手法を提案する。
    \item このデータセットを作成し、それをオープンソースのText2SQLモデルを作成した。このモデル(OmniSQL)は、既存のオープンソースモデルやクローズドモデルと同等もしくはそれ以上の性能を示した。
\end{itemize}
\end{simplebox}

\section{手法}
\begin{method}{データ合成フレームワーク}
\begin{itemize}
    \item 説明A
    \item 説明B
\end{itemize}
\end{method}

\begin{method}{データ合成フレームワークで作成したSQLの統計情報}
\begin{itemize}
    \item .
\end{itemize}
\end{method}

\begin{method}{OmniSQL: Text2SQLモデル}
\begin{itemize}
    \item テーブルスキーマ生成
    \begin{itemize}
        \item Web上にある表形式データから、「テーブル名、テーブルの説明、列名、列のデータ型、列の説明、例示データ」を生成する。
        \item 上のように生成したテーブルは単純すぎる傾向があり、また主キーなどのリレーションが不完全であることがあるため、テーブルに関連する列を追加することでスキーマ定義の一貫性を改善する。
    \end{itemize}
    \item SQLクエリ生成
    \begin{itemize}
        \item SQLクエリをLLMで生成するが、大規模なLLMは過度に複雑なクエリを生成する傾向があるため、クエリのレベルを設定し、指定されたレベルに対応するSQLクエリを生成するようLLMに指示する。具体的には「タスク指示、スキーマ、SQL関数、データベースの値、SQLの複雑性、アウトプットのカラムの数」を与えてSQLを生成させる。
        \item そのあとに生成されたクエリのうちSELECTクエリのみ抽出して実行し、成功するもののみを採用する。
    \end{itemize}
    \item SQLクエリから自然言語の質問文生成を行う
    \begin{itemize}
        \item 
    \end{itemize}
    \item CoT解法の生成
    \begin{itemize}
        \item 
    \end{itemize}
\end{itemize}
\end{method}

\section{実験結果}
\begin{experiment}{実験手法}
\begin{itemize}
    \item 手法A.
\end{itemize}
\end{experiment}

\begin{experiment}{実験結果}
\begin{itemize}
    \item 結果A
    \item 結果B
\end{itemize}
\end{experiment}

\section{感想}
\begin{kansou}
\begin{itemize}
  \item 感想A
  \item \ref{fig:template}のように図を引用することができる.
\end{itemize}
\end{kansou}

\begin{figure}
    \centering
    \includegraphics[width=0.5\textwidth]{img/image.png}
    \caption{キャプション}
    \label{fig:template}
\end{figure}

\bibliographystyle{jplain}
\bibliography{template.bib}

\end{document}