\documentclass[dvipdfmx,uplatex]{jsarticle}

%% Packages
\usepackage{graphicx,color,hyperref}
\usepackage{algorithm}
\usepackage{algorithmic}
\usepackage{url}
\usepackage{lscape}
\usepackage{mathtools}
\usepackage{here}
\usepackage{amsmath,amssymb,amsfonts}
\usepackage{amsthm}
\usepackage{tikz}
\usepackage{tcolorbox}
\usepackage{pxjahyper}

%% Theorem Styles
\newtheorem{theorem}{定理}
\newtheorem{proposition}{命題}
\newtheorem{cor}{系}
\newtheorem{definition}{定義}
\newtheorem{problem}{問題}
\theoremstyle{remark}
\newtheorem{remark}{注意}
\newtheorem{requirement}{条件}

%% Environment (Colorful Box)
\newenvironment{simplebox}{
    \begin{tcolorbox}[
        fonttitle=\bfseries,
    ]
}{
    \end{tcolorbox}
}

\newenvironment{method}[1]{
    \begin{tcolorbox}[
        colframe=green!50!black,
        colback=green!50!black!10!white,
        colbacktitle=green!50!black!40!white,
        coltitle=black,
        fonttitle=\bfseries,
        title={#1}
    ]
}{
    \end{tcolorbox}
}

\newenvironment{experiment}[1]{
    \begin{tcolorbox}[
        colframe=violet,
        colback=violet!10!white,
        colbacktitle=violet!40!white,
        coltitle=black,
        fonttitle=\bfseries,
        title={#1}
    ]
}{
    \end{tcolorbox}
}

\newenvironment{kansou}{
    \begin{tcolorbox}[
        colframe=brown,
        colback=brown!10!white,
        colbacktitle=brown!40!white,
        coltitle=black,fonttitle=\bfseries
    ]
}{
    \end{tcolorbox}
}

%% Title
\title{The dataflow model: a practical approach to balancing correctness, latency, and cost in massive-scale, unbounded, out-of-order data processing}
\author{\empty}
\date{\empty}

%% Document body
\begin{document}
\maketitle

\begin{itemize}
    \item Link: \url{https://dl.acm.org/doi/10.14778/2824032.2824076}
    \item Conference: VLDB2015
    \item Citation: \cite{dataflow-model}
    \item Content: \url{https://www.vldb.org/pvldb/vol8/p1792-Akidau.pdf}
\end{itemize}

\section{概要}
\begin{simplebox}
\begin{itemize}
    \item 概要A
    \item 概要B
\end{itemize}
\end{simplebox}

\section{背景}
\begin{simplebox}
\begin{itemize}
    \item 非有界 vs 有界
    \begin{itemize}
        \item 非有界なデータセットとは無限に続いていくデータセットのこと.
        \item 有界なデータセットとは、有限のデータセットのこと.
    \end{itemize}
    \item ウィンドウ処理: データセットを時間で分割して処理すること.
    \begin{itemize}
        \item Fixed window: 長さが固定されていて、時間で分割されるウィンドウ (e.g. 1時間ごと, 1日ごと)
        \item Sliding window: ウィンドウサイズとスライド周期で定義されるウィンドウ (e.g. 30分のウィンドウを1時間ごとに開始)
        \item Session window: データのあるキーごとにウィンドウを作成する (e.g. ユーザごと1時間の間に起こした行動を集めるウィンドウ)
    \end{itemize}
    \item 時間の概念
    \begin{itemize}
        \item イベント時間: イベントそのものが発生した時間
        \item 処理時間: あるイベントがパイプラインの中で処理された時間
    \end{itemize}
    \item 理想的にはイベント時間と処理時間は一致するが、現実にはイベント時間と処理時間はずれる.
\end{itemize}
\end{simplebox}

\section{手法}
\begin{method}{Dataflow Model}
\begin{itemize}
    \item 構成要素
    \begin{itemize}
        \item ParDo: 各入力要素に対してユーザ定義関数が適用されて、1つの入力から0個以上の出力を生成する.
        \item GroupByKey: (key, value)のペアをkeyごとにグループ化する. このときに無限データを仮定すると全データの収集は不可能となる. そこでウィンドウ処理が必要となる.
    \end{itemize}
    \item ウィンドウ処理
    \begin{itemize}
        \item 実質的にGroupByKeyをGroupByKeyAndWindowとして再定義する.
        \item イベント時間に基づくウィンドウ処理をサポートするために(key, value, event\_time, window)のタプルを一つのデータとして扱う. この形で1. ウィンドウ割り当てと2. ウィンドウマージをできるようにすることで任意のウィンドウ操作が可能になる.
        \begin{itemize}
            \item ウィンドウ割り当て: windowを設定すること、同じデータを二つ上のウィンドウに割り当てるときは複製する.
            \item ウィンドウマージ: マージする基準に従って、データをマージして新しいデータを作る.
        \end{itemize}
    \end{itemize}
    \item トリガー
    \begin{itemize}
        \item 非有界なデータにおいては、いつウィンドウを閉じるか(ウィンドウに対する処理を完了するか)という問題があり、ウィンドウの完了を示す別のシグナルが必要となる.
        \item Lambdaアーキテクチャでは、ストリーミング処理中はレイテンシが小さい近似値を出力し、バッチ処理で正確な値を出力する.
        \item これを参考にして、一つのウィンドウが複数回の出力を行える仕組みを作る; これをトリガーとよぶ.
        \item たとえば、データ件数が$N$件に達したらトリガーを発火させる、または、イベント時間がある時刻に達したらトリガーを発火させるなど.
    \end{itemize}
    \item Pane (おそらく出力のこと): ウィンドウの出力はトリガーによって発火させる, すなわち複数回出力するため、その出力をどのように扱うかを決める必要がある.
    \begin{itemize}
        \item Discarding: トリガー時にウィンドウの中身が破棄されて、後続の結果は過去の結果と一切関係を持たない.
        \item Accumulating: トリガー時にウィンドウの中身は残って、次の結果は前の結果を上書きする.
        \item Accumulating and Retracting: 出力済みの値を記録しておき、次のトリガー発火時に前の値を取り消す信号を最初に出力し、そのあとに新しい値を出力する.
    \end{itemize}
    \item Example
    \begin{itemize}
        \item TODO
    \end{itemize}
\end{itemize}
\end{method}

\begin{method}{Implementation and Design}
\begin{itemize}
    \item TODO
\end{itemize}
\end{method}

% \section{実験結果}
% \begin{experiment}{実験手法}
% \begin{itemize}
%     \item 手法A.
% \end{itemize}
% \end{experiment}

% \begin{experiment}{実験結果}
% \begin{itemize}
%     \item 結果A
%     \item 結果B
% \end{itemize}
% \end{experiment}

\section{感想}
\begin{kansou}
\begin{itemize}
  \item Apach Beamのベースとなったデータフローモデルを提案した論文.
  \item \ref{fig:template}のように図を引用することができる.
\end{itemize}
\end{kansou}

% \begin{figure}
%     \centering
%     \includegraphics[width=0.5\textwidth]{img/image.png}
%     \caption{キャプション}
%     \label{fig:template}
% \end{figure}

\bibliographystyle{jplain}
\bibliography{template.bib}

\end{document}