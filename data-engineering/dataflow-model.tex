\documentclass[dvipdfmx,uplatex]{jsarticle}

%% Packages
\usepackage{graphicx,color,hyperref}
\usepackage{algorithm}
\usepackage{algorithmic}
\usepackage{url}
\usepackage{lscape}
\usepackage{mathtools}
\usepackage{here}
\usepackage{amsmath,amssymb,amsfonts}
\usepackage{amsthm}
\usepackage{tikz}
\usepackage{tcolorbox}
\usepackage{pxjahyper}

%% Theorem Styles
\newtheorem{theorem}{定理}
\newtheorem{proposition}{命題}
\newtheorem{cor}{系}
\newtheorem{definition}{定義}
\newtheorem{problem}{問題}
\theoremstyle{remark}
\newtheorem{remark}{注意}
\newtheorem{requirement}{条件}

%% Environment (Colorful Box)
\newenvironment{simplebox}{
    \begin{tcolorbox}[
        fonttitle=\bfseries,
    ]
}{
    \end{tcolorbox}
}

\newenvironment{method}[1]{
    \begin{tcolorbox}[
        colframe=green!50!black,
        colback=green!50!black!10!white,
        colbacktitle=green!50!black!40!white,
        coltitle=black,
        fonttitle=\bfseries,
        title={#1}
    ]
}{
    \end{tcolorbox}
}

\newenvironment{experiment}[1]{
    \begin{tcolorbox}[
        colframe=violet,
        colback=violet!10!white,
        colbacktitle=violet!40!white,
        coltitle=black,
        fonttitle=\bfseries,
        title={#1}
    ]
}{
    \end{tcolorbox}
}

\newenvironment{kansou}{
    \begin{tcolorbox}[
        colframe=brown,
        colback=brown!10!white,
        colbacktitle=brown!40!white,
        coltitle=black,fonttitle=\bfseries
    ]
}{
    \end{tcolorbox}
}

%% Title
\title{The dataflow model: a practical approach to balancing correctness, latency, and cost in massive-scale, unbounded, out-of-order data processing}
\author{\empty}
\date{\empty}

%% Document body
\begin{document}
\maketitle

\begin{itemize}
    \item Link: \url{https://dl.acm.org/doi/10.14778/2824032.2824076}
    \item Conference: VLDB2015
    \item Citation: \cite{dataflow-model}
    \item Content: \url{https://static.googleusercontent.com/media/research.google.com/en//pubs/archive/43864.pdf}
\end{itemize}

\section{概要}
\begin{simplebox}
\begin{itemize}
    \item 概要A
    \item 概要B
\end{itemize}
\end{simplebox}

\section{背景}
\begin{simplebox}
\begin{itemize}
    \item 背景A
    \item 背景B
\end{itemize}
\end{simplebox}

\section{手法}
\begin{method}{Jacobi Decoding}
\begin{itemize}
    \item 説明A
    \item 説明B
\end{itemize}
\end{method}

\begin{method}{Lookahead Decoding}
\begin{itemize}
    \item 説明A
    \item \cite{dataflow-model}のように引用できる.
\end{itemize}
\end{method}

\section{実験結果}
\begin{experiment}{実験手法}
\begin{itemize}
    \item 手法A.
\end{itemize}
\end{experiment}

\begin{experiment}{実験結果}
\begin{itemize}
    \item 結果A
    \item 結果B
\end{itemize}
\end{experiment}

\section{感想}
\begin{kansou}
\begin{itemize}
  \item 感想A
  \item \ref{fig:template}のように図を引用することができる.
\end{itemize}
\end{kansou}

% \begin{figure}
%     \centering
%     \includegraphics[width=0.5\textwidth]{img/image.png}
%     \caption{キャプション}
%     \label{fig:template}
% \end{figure}

\bibliographystyle{jplain}
\bibliography{template.bib}

\end{document}