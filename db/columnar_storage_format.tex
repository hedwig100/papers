\documentclass[uplatex]{jsarticle}

%% Packages
\usepackage[dvipdfmx]{graphicx,color,hyperref}
\usepackage{algorithm}
\usepackage{algorithmic}
\usepackage{url}
\usepackage{lscape}
\usepackage{mathtools}
\usepackage{here}
\usepackage{amsmath,amssymb,amsfonts}
\usepackage{amsthm}
\usepackage{pxjahyper}

%% Theorem Styles
\newtheorem{theorem}{定理}
\newtheorem{proposition}{命題}
\newtheorem{cor}{系}
\newtheorem{definition}{定義}
\newtheorem{problem}{問題}
\theoremstyle{remark}
\newtheorem{remark}{注意}
\newtheorem{requirement}{条件}

%% Title
\title{An Empirical Evaluation of Columnar Storage Formats}
\author{\empty}
\date{\empty}

%% Document body
\begin{document}
\maketitle

\begin{itemize}
    \item Link: \url{https://dl.acm.org/doi/10.14778/3626292.3626298}
    \item Conference: VLDB2024
    \item Citation: \cite{columnar_storage_formats}
    \item Arxiv: \url{https://arxiv.org/abs/2304.05028}
\end{itemize}

\section{概要}

\section{手法}
\cite{mypaper}すごい
\section{実験}

\begin{figure}
  \centering
  \includegraphics[width=0.5\textwidth]{img/image.png}
  \caption{キャプション}
  \label{fig:template}
\end{figure}

\section{感想}
\ref{fig:template}のように図を引用することができる.


\bibliographystyle{jplain}
\bibliography{template.bib}

\end{document}